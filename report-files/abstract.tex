\documentclass[a4paper, 12pt]{article}
\usepackage{graphicx}
\usepackage{biblatex}
\addbibresource{citations.bib}

\begin{document}
\begin{titlepage}
    \begin{center}
        \vspace*{1cm}

        \huge
        \textbf{matsubara\\ (EXTENDED ABSTRACT)}

        \vspace{1cm}

        \textit{Alexander Stradnic}

        \vspace{3cm}

        \includegraphics[width=0.6\textwidth]{"piano.jpg"}

        \vfill

        \textsf{Final Year Project on the Use of Sequential Ordering for Creating Playlists}

        \vspace{1cm}

        \textbf{University College Cork}

        \vspace*{1cm}

    \end{center}
\end{titlepage}

\paragraph{Abstract}
In this Final Year Project I present the use of Case-based Recommendation in creating music playlists, 
along with providing an implementation of this CBR system.
Playlists are sequences of songs arranged in a particular order. Using this information, 
context can be gained from previous playlists in order to build a new playlist from a given seed song or starting list.
This project is based on the below papers \cite{1, 2}.



\section{Introduction}
My project is on creating playlists using Recommender Systems, focusing on using Case-based Sequential Ordering. 

\paragraph{Case-based Recommendation(CBR):}
The approach in which a playlist is generated from a seed song/shorter playlist using sequential patterns learned from 
a dataset of existing playlists.


CBR is a method of recommendation which can be used in contexts in which a meaningful order or sequence to objects is desired or useful.
There can also be a large variety of possible values (such as songs in this case).
A set of playlists are then selected from the total dataset based on relevance and other criteria forming the Case Base.
From this Case Base of playlists \(\mathcal{C}\), two main values are computed for each playlist:
\begin{enumerate}
    \item The Attribute Variety
    \item The Coherence of the Playlist
\end{enumerate}

\subparagraph*{Variety}
The variety of each playlist in \(\mathcal{C}\) is calculated based on the repetition of song features within a set distance.
\subparagraph*{Coherence}
The coherence of each playlist in \(\mathcal{C}\) is calculated based on how related it is to the input song. 

\section{Datasets}
The data was acquired from Spotify's Million Playlist Dataset.
Processing was done on the dataset running in SQLite.
The data was downloaded from Spotify's servers, structured as a set of JSON files.
This was then converted into an SQLite database to improve performance and maintain consistency the with earlier similarity-based 
algorithms which used a shortened collection from MusicBrainz called the Million Song Dataset.

The processing itself is done using \emph{Python}. This is due to the amount of useful libraries such as \emph{Pandas} and \emph{Numpy}. 

\section{Playlist Generation using CBR}


\section{Testing}
A variety of CBR systems will be tested with varying weightings of Coherence and Variance, along with simpler pseudorandom and similarity-based algorithms.
These will then be ranked by survey.

\printbibliography

\end{document}
